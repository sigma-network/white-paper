\section{Entropy collection}
\label{entropy}

When a validator creates a new account, it is important that they not be able to predict its ticket scores. Otherwise, a validator could enumerate possible key pairs until they eventually find one with high-scoring tickets. To prevent such ``grinding'' attacks, we use a new random seed in each epoch. A validator account created in epoch $i$ cannot submit tickets until epoch $i + 2$, so there is a full round of entropy generation in epoch $i + 1$ before the account becomes eligible.

The random seed for each epoch is derived from the signatures of block creators. In particular, each block creator submits $\operatorname{R}(\mathrm{sk}, h)$ as their entropy contribution, where $R$ is a pseudorandom function, sk is their secret key and $h$ is the height of the block. This computation is done in zero knowledge; the details are described in \autoref{block-creation-circuit}. At the end of each epoch, the bitwise XOR of these entropy submissions becomes the random seed of the following epoch.

Note that this scheme has the properties of a verifiable random function (VRF), although it does not use a typical VRF construction. In particular, the pseudorandom computation is deterministic, and its correctness is publically verifiable since a zero-knowledge proof is provided. Therefore, the only way for a block creator to influence the entropy stream is to skip their block, forfeiting the block reward.

\subsection{Profitability of entropy manipulation}

Consider a validator with a stake of $p$, expressed as a fraction of the money supply that is actively staked. Let $b$ be the number of contiguous blocks at the very end of an epoch which they are eligible to create. By submitting or withholding blocks, the validator can manipulate up to $b$ bits of entropy. In other words, they may choose from among $2^b$ random seeds for the next epoch.

Net $n$ be the number of blocks in an epoch. Let $r_1$ be the validator's representation in the next epoch (that is, the number of blocks they will be eligible to create) which would result from the ``default'' seed, if no manipulation occurs. Let $r_2, \dots, r_{2^b}$ be the validator's representation resulting from the alternative seeds which the validator can choose from.

If the validator performs no entropy manipulation, their expected block rewards from the next epoch would simply be $n p$. We will compare this to the expected block rewards from a manipulative validator, who calculates their block rewards for all possible seeds, and choses the seed which results in them receiving the most block rewards. In other words, we want to calculate

$$ E\left[ \max_{1 \le i \le 2^b}(r_i) \right] $$

Note that this is an upper bound, because we are not accounting for any missed block rewards which a manipulator would incur from withholding blocks.

To calculate this expected value, we start by applying Fubini's theorem with the linearity of expectation:
\begin{align*}
  E\left[ \max_{1 \le i \le 2^b}(r_i) \right] &= \sum_{j=0}^{n} P\left( \max_{1 \le i \le 2^b}(r_i) > j \right) \\
  &= \sum_{j=0}^{n} P(r_1 > j \text{ or } \dots \text { or } r_{2^b} > j) \\
  % Alternate version:
  %\intertext{We now apply the principle of inclusion-exclusion. Since $r_1, \dots, r_{2^b}$ are identically distributed, we use $r_1$ in place of the other variables, giving}
  %&= \sum_{j=0}^{\infty} \sum_{k=1}^{2^b} \left( (-1)^{k+1} P(r_1 > r_1 + j)^k \right) \\
  %\intertext{Substituting in the binomial PDF, we have}
  %&= \sum_{j=0}^{\infty} \sum_{k=1}^{2^b} \left( (-1)^{k+1} \sum_{l=r_1 + j + 1}^{n} \left( \binom{n}{l} p^l (1-p)^{n-l} \right) \right) \\
  %&= \sum_{k=1}^{2^b} \left( (-1)^{k+1} \sum_{j=0}^{\infty} \sum_{l=r_1 + j + 1}^{n} \left( \binom{n}{l} p^l (1-p)^{n-l} \right) \right) \\
  %&= \sum_{k=1}^{2^b} \left( (-1)^{k+1} \sum_{l=r_1 + j + 1}^{n} \left( (j + 1) \binom{n}{l} p^l (1-p)^{n-l} \right) \right) \\
  \intertext{Applying De Morgan's law, this becomes}
  &= \sum_{j=0}^{n} \left( 1 - P(r_1 \le j \text{ and } \dots \text { and } r_{2^b} \le j) \right) \\
  \intertext{Since $r_1, \dots, r_{2^b}$ are independent and identically distributed, we can separate the probabilities and use $r_1$ in place of the other variables:}
  &= \sum_{j=0}^{n} \left( 1 - P(r_1 \le j)^{2^b} \right) \\
  \intertext{Finally, applying the binomial CDF, we have}
  &= \sum_{j=0}^{n} \left( 1 - \left( \sum_{k=0}^{j} \binom{n}{k} p^k (1-p)^{n-k} \right)^{2^b} \right)
\end{align*}
where $n$ is the number of blocks in an epoch.

We now have a formula for the validator's expected advantage given a particular number of manipulable bits $b$. But since $b$ varies, we want to study the expected advantage for all possible values of $b$, weighted by their probabilities:
\begin{align*}
  E\left[ a \right] &= \sum_{b=0}^{n} P(b) E\left[ a \mathbin{\vert} b \right] \\
  \intertext{Applying the formula we derived above, along with $P(b) = p^b$, we get}
  &= \sum_{b=0}^{n} \left( p^b \sum_{j=0}^{n} \left( 1 - \left( \sum_{k=0}^{j} \binom{n}{k} p^k (1-p)^{n-k} \right)^{2^b} \right) \right)
\end{align*}
Finally, to compute the expected advantage from manipulation, $E[a]$, we simply divide this quantity by the expected rewards with no manipulation, which is $n p$.

\begin{wrapfigure}{r}{.42\textwidth}
  \begin{tabularx}{.42\textwidth}{|Y|Y|Y|}
    \hline
    $p$ & $n$ & $E[a]$ \\
    \hline
    \multirow{3}{*}{25\%}
    & 100 & 3.09\% \\
    \cline{2-3}
    & 1000 & 0.98\% \\
    \cline{2-3}
    & 10000 & 0.31\% \\
    \hline
  \end{tabularx}
  \caption{The expected advantage resulting from manipulation.}
  \label{fig:manipulation}
\end{wrapfigure}

The results are shown in \autoref{fig:manipulation}. [TODO: Add some results with different p values, and some commentary, noting that actual advantages will be even lower because of imperfect information and the missed block rewards we didn't account for.]

\subsection{Alternative designs}

There are several potential ways to increase the penalty for entropy manipulation. One option is to require that validators make a deposit before contributing entropy, and slash the deposit of any validator who withholds their entropy contribution. This is essentially the idea behind RANDAO \cite{randao}. RANDAO also uses a commit-reveal mechanic, but this is unnecessary with a VRF scheme like ours, where entropy contributions are computed deterministically. [TODO: Finish this. Mention BLS and VDF schemes.]

%[Also considering verifiable delay functions which can provide manipulation-proof entropy, but it may be best to wait for new developments in this very new field. \cite{wesolowski2018slow} seems practical, but relies on discrete log hardness.]
