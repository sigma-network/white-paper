\section{Background}
\label{background}

The success of Bitcoin demonstrates a strong interest in decentralized currencies, where monetary policies are determined by a community of users rather than government decree. Bitcoin's purported anonymity contributed to its success, but a growing body of research shows that accounts can often be deanonymized based on their public activity \cite{meiklejohn2013fistful} \cite{androulaki2013evaluating}.

Some newer cryptocurrencies improve upon Bitcoin's privacy. Monero \cite{van2013cryptonote} uses ring signatures to obfuscate a transaction's source of funds, and Diffie-Hellman exchanges to hide its destination. Similarly, Zcash \cite{sasson2014zerocash} uses zk-SNARKs to prove that a transaction has a valid source of funds without revealing what that source is. Zcash's privacy guarantee is stronger by comparison, since its shielded transactions leak no information at all, whereas Monero uses rings of size 7 by default.

While Zcash and Monero offer strong privacy, both protocols are designed to be executed by individual computers, so their transaction volume is limited by the capabilities of a single machine. Both protocols have sufficient capacity for the current transaction volume, but their limited scalability precludes their adoption in mainstream commerce.

Another drawback of Bitcoin, Zcash and Monero is that they rely on proof of work for consensus. The security model fundamentally requires that mining be expensive, as this creates an economic barrier to 51\% attacks. At the time of writing, Bitcoin mining is funded by about \$4 billion worth of block rewards per year, which causes an inflation rate of about 4\%.% As the block reward decreases, inflation will decrease, but transaction fees will need to increase in order to maintain the same security level. In one way or another, Bitcoin users must fund mining operations in order to maintain adequate security.

Yet another challenge is latency. Bitcoin has an average block time of 10 minutes, and \cite{nakamoto2008bitcoin} recommends waiting for six confirmations before accepting a transaction as final. Some newer cryptocurrencies such as Tendermint \cite{kwon2014tendermint} and Algorand \cite{chen2018algorand} provide quick finality using BFT consensus algorithms, but they are subject to the same privacy issues as Bitcoin.

Finally, most existing cryptocurrencies do not offer post-quantum security. This includes Zcash and Monero, both of which rely on the hardness of the discrete logarithm problem for transaction soundness. A universal quantum computer could compute discrete logarithms in polynomial time using a variant of Shor's algorithm. To date, Shor's algorithm has been demonstrated only with very small factorization problems \cite{martin2012experimental}. Still, the potential future impact of quantum computers is a cause for concern.

Our design aims to address all of the aforementioned challenges. It offers a strong privacy guarantee using zero knowledge proofs, similar to Zcash, but using a post-quantum secure proof system based on multiparty computation. Our blockchain protocol is based on proof of stake, so security is inexpensive compared to proof of work blockchains. We use use a novel BFT consensus algorithm to quickly finalize each block, thereby ensuring immutability and eliminating forks. To allow transaction volume to scale, we use a simple sharding mechanism which lets each validator divide their communications among multiple servers.
